\section{CIVIS: A Community-Oriented Design in Future Smart Grids}

\begin{svgraybox}
Discuss the CIVIS project making relation to the previous theory section. 

The discussion shall not be limited to the app YouPower, but also the other efforts made around it (if they are related to the discussions in the previous section), e.g. the user stories, focus groups workshops, interviews, participatory budgeting, etc. 

Use the YouPower paper as much as possible. 
\end{svgraybox}


\subsection{Understanding and Formulation of the Design Situation}
\begin{svgraybox}
[note by GP] Giacomo can write this part. To be expanded/integrated by all.

It will focus on CIVIS project by outlining its foundations (as fp7 R\&I project) and approach (interaction with local stakeholders + incremental design \& development of platform architecture). It will describe the inital, explicit objectives (+ simple rationale/roadmapp to achieve them) and it will provide an overview of the local contexts (from social, technical and energy point of view), in order to make clearer who are the involved stakeholders and their context.
\end{svgraybox}

Since more than two decades the ongoing and long-term energy transition process is shifting the energy domain
towards decentralization, distributed production and renewable sources \cite{rifkin_third_2011; sovacool_how_2016}.
Several general and intertwined aspects contribute to this transition: (\textit{i}) the awareness of the inherent complexities
that exist among energy systems, societies and the environment \cite{bulkeley_bringing_2012; umbach_global_2010}; (\textit{ii}) the
widespread diffusion of new, enhanced technologies and their hybridization with contemporary ICTs \cite{putrus_smart_2013; schick_innovating_2013};
(\textit{iii}) the pursuit of national and supranational energy policies around energy efficiency, sustainability and low carbon emissions \cite{da_graca_carvalho_eu_2012};
and (\textit{iv}) the emergence of new actors in the energy value chain, such as energy cooperatives and
energy communities \cite{viardot_role_2013}, or the transformation of old ones, such as housing associations,
and amateur energy managers \cite{hasselqvist_linking_2016}.

CIVIS work took place under European Union’s interest to foster energy transition by tackling the so called societal challenge of efficient energy. 
This was pursued the vision of a smart grid and, basically, the use of ICTs as main driver for the
reconfiguration of relationships among traditional and emerging actors in the energy value chain – \textit{i.e.} distributors, producers, retailers and prosumers, cooperatives.
In particular, CIVIS was a three year, EU project\footnote{\url{http://cordis.europa.eu/project/rcn/110429\_en.html}} funded under the \textit{FP7 Smart Cities} framework, that pursued the design,
prototyping and real-life testing of a platform for the improvement of energy behaviours in the domestic sector. The project was structured around three main areas
of interest – \textit{i.e.} energy, ICT, social innovation and business – and organized into three broad phases that roughly
overlapped with the project years and that ensured a close interaction with the local realities and contexts of
the pilots: (\textit{i}) an exploratory phase, used to align CIVIS’ overarching objectives with the local contexts’ needs;
(\textit{ii}) a prototyping one, which concerned the actual design and development of the platform (from data monitoring devices to the
front-end applications); and (\textit{iii}) a final testing phase which included the full scale deployment of the platform in the pilots for usage and assessment purposes.

Italy and Sweden hosted two pilots each. 
The two municipality areas of Storo and San Lorenzo, in Trentino Alto-Adige (a region in north-west Italy), included the
Italian pilots. Here, two electric cooperatives, producing and selling 100\% renewable energy
to their associate members, together with two samples of recruited associate member households acted as the main stakeholders.
The regional distribution system operator (DSO), the institutional representatives of the two municipalities and two local cultural associations participated as
stakeholders in various phases of the project, by providing knowledge and support for technical aspects related
to energy and households engagement.
The area of Stockholm hosted the two Swedish pilots. One involved the residential and central neighbourhood
of Hammarby Sj\"{o}stad, which included apartment buildings owned by housing cooperatives\footnote{In Sweden, those who buy a home officially own the right to inhabit the estate and must
join a corresponding \textit{housing cooperative} that owns and maintains the estates. The members of a cooperative
annually elect a board that makes energy related decisions on behalf of the members.}.
Recruited households from the cooperatives and cooperatives' board members acted
as key stakeholders here. \textbf{\textit{[NOTE: Do we want to include Fardala?]}} The other pilot concerned a townhouse area in the outskirts of Stockholm: F\aa{}rdala. In this townhouse
area the local residents’ association and some of its member households participated to CIVIS.


\subsection{Products of the Design Process}

TODO

\paragraph{Evaluation} After a successful deployment, YouPower was evaluated in the test sites in Sweden and Italy. We collected in parallel, the data on user engagement with the app and on energy prosumption. The initial evaluation revealed how the Trentino residents engaged with a weather feature that predicts the solar energy production levels, while the Stockholm housing association managers used the community features to engage with the residents and connect with other managers. 

Some of the tangible results in the Italian test site include the savings in the electricity and heating energy spent or produced \cite{civics}:
\begin{itemize}
	\setlength{\itemsep}{0mm}
	\item percent	of	self-consumption	of	 the	PV	self-produced energy	 is	 increased	 for more than	50\% of the	users	comparing	to	the period before CIVIS;
	\item electricity	consumption	from the	grid	is	reduced for the same period for	more than	50\% of the	consumers;	
	\item 	total	 electricity	 consumption	 (including both, from	the grid	 and	the PV	 self-consumption)		is	reduced for the same period for	more than	50\% of the	consumers;	
	\item  the users spent less than 11\% of hours in overheating their spaces.\footnote{The	heating behaviour	 of	 CIVIS	 users (control	 of	 space	 heating),	 met	 the	recommended	standard	values.}
\end{itemize}

The housing cooperatives (BRF; bostadsr\"attsf\"orening	in	Swedish) in Stockholm conducted following actions as a result of the YouPower use \cite{civics}:

\begin{itemize}
	\setlength{\itemsep}{0mm}
	\item BRF	Grynnan: adjustments of	ventilation	system and	turning	off	the	outdoor	ice	melting	system (resulting in	reductions	in	heating);
	\item BRF	 Sj\"ostaden	 1: extra	 insulation	 added	 to	 the	 roofs in	May	2015	(lead	to	a	decrease	in	heating	consumption);	 installation	 of	 heat	 recovery	 heat	 pumps	 in	 February	 2016 (lead	 to	 some	 increase	 in	 electricity	 consumption	 but	 the	 overall	savings);
	\item BRF	\"Alven: ventilation	optimization due to lower thermal comfort,	however	the	energy	use	went	up;
	\item BRF	 Seglatsen: installation of recovery	heat	pumps	in	 the	building (reduction in	 the	heating	consumption	by	around	60\%, and since electrcitiy consumption increased, overall savings are	about	40\%).	
	\item BRF	 Hammarby	 Kanal:	 ventilation optimization;
	\item BRF	 Hammarby	 Ekbacke: a goal	 based	 energy	 reduction in	a	new	business	model	for	the	housing	associations	where	 they do not	 have	 to	 pay upfront	costs	for	the energy 	actions, and then part	of	the	savings	goes to	the	 ESCO (European Skills, Competences, Qualifications and Occupations)	 for	 a	 fixed	 time	 period.	
\end{itemize}


To assess the long term effectiveness of the CIVIS social energy intervention,  after a certain period of time, it would be beneficial to fuse the collected data from the app with those about the consumption/prosumption. 

\paragraph{Limitations} The initial lack of data due to acquisition and privacy issues in the residential setting was among the main limitation in this project. We installed the IoT sensors to collect additionally needed data for using the introduced app. 

\subsection{Design Process (or participatory design?)}

\begin{svgraybox}
[note by GP] Giacomo can write this part. 

\end{svgraybox}


\section{Discussions}

\begin{svgraybox}
Lessons Learned?  / Design Guidelines? 
\end{svgraybox}