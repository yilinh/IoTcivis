\section{CIVIS: A Community-Oriented Design in Future Smart Grids}
\label{sec:civis}

This section presents the EU CIVIS project\footnote{\url{http://cordis.europa.eu/project/rcn/110429\_en.html}} as an illustration of designing STS.  The project took place under EU's interest to address the societal challenges of energy efficiency. The vision of smart grids and the use of IoT and ICT are 
the main drivers for the project's ambition to reconfigure the relationships among traditional and 
emerging actors -- producers, distributors, retailers, 
prosumers and cooperatives -- in the energy value chain.
In the following, we review the literature about IoT with regard to smart grids and STS,
provide an overview of the design situation, and then discuss
the collaborative design process and the main outcomes.  

\subsection{Internet-of-Things and Smart Grids as Socio-technical Systems}
\label{sec:IoT_socio-techical}


%Since 1999, when Ashton \cite{ashton2011internet} coined the term \textbf{Internet of Things (IoT)}, while he was introducing RFID technology in the context of supply-chain management, the meaning of the term has evolved. Today, 

The International Telecommunication Union defines IoT as the worldwide network of interconnected objects uniquely addressable based on standard communication protocols -- a definition focusing on the technological aspect of IoT. 
%From the technical aspect, the IoT can be divided into three following layers:
%\begin{itemize}
%	\item comprehensive sensing (perception layer),
%	\item reliable transmission (network layer), and
%	\item intelligent processing (application layer).
%\end{itemize}
Since IoT is expected to have a massive impact on society and wider cultural milieu, its ultimate status should accordingly be a human-centred STS although how the IoT landscape will look like in the future is yet uncertain \cite{atzori2014smart,guo2012opportunistic,ning2011future,Shin2014,tomasini2017effect}. 
%
%Nevertheless, the envisioned IoT represents a , instead of an only technical system, as the interconnected objects are intended to interact also with people and society \cite{atzori2014smart,Shin2014}. 
%However, the focus among engineers and computer scientists when discussing IoT was mainly on the technical side, as is the case with other socio-technical systems.
%Interestingly, even some of the ideas for the \textit{Social IoT (SIoT)} \cite{atzori2012social,guinard2010sharing,atzori2014smart} discuss the convergence of social networks with IoT mainly with the intention to optimize the interactions among the IoT objects. There is often very little attention placed in such studies on the interrelationship of IoT with the human psychological and social aspects \cite{atzori2012social,guinard2010sharing}. 
%
%The discussion on IoT should also cover another term that is often used interchangeably -- \textbf{ubiquitous computing}, coined by Weiser \cite{weiser1991computer}. Ubiquitous computing is defined as ``the physical world that is richly and invisibly interwoven with sensors, actuators, displays, and computational elements, embedded seamlessly in the everyday objects of our lives, and connected through a continuous network'' (ibid.). In other words, while the Internet has led to the interconnection of people at an unprecedented scale, the IoT is expected to interconnect also the objects around us, leading to a smart environment \cite{gubbi2013internet}. %We see that researches talking about IoT mainly describe connected devices, while ubiquitous computing focuses on the smart environment in which computing is pervasive. 
%Hence, the two terms take a different stance and focus on different aspects of what is envisioned to become the \textit{Future Internet}. 
%The IoT vision put forward by Weiser \cite{weiser1991computer} led to a fruitful new field within computer science (ubiquitous computing). In particular, the original vision suggested that ubiquitous computing can lead to an environment that is predicting and adapting to the people's needs, while the people were considered passive elements (i.e., technological determinism). However, Rogers \cite{rogers2006moving} offered a constructive critique of this vision. Namely, Rogers argued that we should move from a \textit{computing approach} (technical aspects) to a \textit{human approach} (social aspects) in developing the smart environment. Rogers argues: ``To make this happen, however, requires moving from a mindset that wants to make the environment smart and proactive to one that enables people, themselves, to be smarter and proactive in their everyday and working practices.''
%
%The effects of the critique, such as the one by Rogers, are seen in researches increasingly adopting the design of IoT as a socio-technical artifact \cite{ning2011future,guo2012opportunistic,Shin2014,tomasini2017effect}. One human-centric vision is illustrated by the \textit{Opportunistic IoT} \cite{guo2012opportunistic,guo2013opportunistic}. The authors explain that
%
%In the light of such ideas, Ortiz et al.\ \cite{ortiz2014cluster} revisit previously introduced ideas of {SIoT} (that still treat IoT only as technical system) and provide a more complete vision of it -- one that treats the IoT as a socio-technical system.
%
%\begin{comment}
%	\sidecaption
%	\includegraphics[scale=.55]{img/opprtunisticIoT}
%	\caption{Human-centric Opportunistic IoT \cite{guo2012opportunistic}}
%	\label{fig:opportunisticIoT} 
%\end{comment}
%
%Some of the key ideas in which the visions of IoT as a socio-technical system (SIoT) expand on those visions that treat it only as a technical system are the following. 
%SIoT visions argue that the efforts should be placed on using and integrating the IoT in society and on the user-level, and building new communities of users and technology while considering actual adoption possibilities (versus designing intrusive technology that will not be used in the end). Moreover, the emphasis is on contextual design, so that IoT will be adapted to different psychological, social, legal, policy and other types of factors. Finally, people have to be empowered to embrace the IoT technologies with awareness raising, proper policies and  human-centered design \cite{ning2011future,Shin2014}.
% 

A key application domain of IoT is envisioned for the smart grids \cite{shin2014socio}. 
%Yun and Yuxin \cite{yun2010research} discuss the possibilities of the IoT to bring about the {smart grid} through sensors, novel telecommunications and computing technologies. 
%The sensors, such as smart, temperature, and illumination meters can collect energy and environmental data. They can also form a high-speed, real-time and bidirectional connection between the consumers, utilities and the electrical grid. Such an improved data collection and communication can support the decision making and in turn improve the overall efficiency of the grid. Interestingly, the technology at the heart of the IoT, the Internet itself, consumes up to 5\% of the total energy spent today in the world. Given the expectation of connecting billions of new devices, this consumption is expected to raise \cite{gubbi2013internet}.	Hence, in addition to its role in optimizing the energy grid, the IoT design itself needs to place accent on sustainability.
%
The IoT technologies can collect energy and environmental data, and form  high-speed real-time bidirectional connections among consumers, utilities and the electrical grid \cite{yun2010research}. The improved data collection and communications can support decision making and in turn improve the overall efficiency of the grid. 
%Interestingly, the technology at the heart of the IoT, the Internet itself, consumes up to 5\% of the total energy spent today in the world. Given the expectation of connecting billions of new devices, this consumption is expected to raise \cite{gubbi2013internet}.	Hence, in addition to its role in optimizing the energy grid, the IoT design itself needs to place accent on sustainability.
%
%Different innovation layers can be used to describe the smart innovation and development characteristics within the smart cities \cite{zygiaris2013smart}. IoT should play an important role in several of those layers: from the interconnection layer with a number of sensors and actuators, through the integration layer monitoring those smart devices, to the intelligent applications layer making use of the real-time data. 
 %IoT can support both the dimensions of demand and supply of energy in the smart grid. 
 IoT is also an integral technology in future smart homes, smart buildings and smart cities  \cite{schatten2014smart,zanella2014internet,zygiaris2013smart}
 %The smart devices that are interconnected and installed in smart buildings, such as smart energy, temperature, and illumination meters are enablers of the smart homes.
%
%IoT can provide transparent energy consumption information of different services in \textit{smart cities}: from lighting, through public transport, to heating and air conditioning of public spaces \cite{zanella2014internet}. %Moreover, the real-time, bidirectional connectivity between the utilities, grid and the users is suggested to lead to the improved overall efficiency of the grid \cite{yun2010research,li2011applications}. 
%Similarly as with IoT in general, we also find visions of smart grid that employ social networks on top of interconnected devices, in this case, the smart meters \cite{ciuciu2012social}. 
% 
where IoT devices are expected to cooperate, actively share energy, and participate in energy management \cite{karnouskos2010cooperative}. 
%It is apparent how in such a context, where IoT meets the smart grid, innovative services and business applications emerge, but also security, privacy and trust gain novel importance.
% Tackling the demand should involve the users \cite{verbong2013smart}. While the focus on technology is still too strong and some smart grid players still perceive the users themselves as the barriers to the smart grid development process, they instead need to understand to what extent the users can act as a solution to the sustainability pathway. 
% 
In addition to object-object interaction, the IoT design must also consider human-object, human-environment and human-human interactions \cite{guo2012opportunistic,guo2013opportunistic}. % taking into account different aspects of human behaviour such as mobility \cite{tomasini2017effect}, preferences \cite{kowshalya2016community} and homophily \cite{atzori2012social}. 
As an ST ensemble, IoT and smart grids should be embedded into society to build new communities of empowered users with an emphasis on contextual design, so that the technologies will be adapted to different psychological, social, legal, policy factors considering actual adoption possibilities (in contrast to designing intrusive technology)  \cite{ning2011future,Shin2014}.

For more than two decades, energy transition has shifted the energy domain
towards decentralization and distributed renewable sources \cite{rifkin_third_2011,sovacool_how_2016}. This transition can be attributed to several intertwined facts: 
(1) the increasing awareness of the inherent complexity among energy systems, societies 
and the environment \cite{bulkeley_bringing_2012,umbach_global_2010}, (2) the
widespread diffusion of new enhanced technologies, such as IoT, and their hybridization with modern ICT
\cite{putrus_smart_2013,schick_innovating_2013}, (3) the pursuit of national and 
supranational energy policies promoting low carbon emission, energy efficiency and sustainability
\cite{da_graca_carvalho_eu_2012}, and (4)~the emergence of new actors such as energy cooperatives and energy communities in the energy value chain \cite{viardot_role_2013}, and the 
transformation of traditional actors such as housing associations and amateur energy managers 
\cite{Hasselqvist2016}.
Under these conditions, many new challenges and possibilities emerge, particularity from an ST perspective  \cite{Shin2014}. 

\subsection{An Overview of the CIVIS Project}

For the CIVIS project, an ST approach was in prospect by design from onset
in the project goal and team composition. The goal in large was to provide ICT support for social participation in smart grids to manage communities and support energy services in the domestic sector. The project team had the ambition to increase citizens' energy awareness, promote environmental values, improve citizens' know-how about sustainable consumption, and to facilitate citizens to improve energy consumption behaviours in their everyday life together with local communities \cite{Huang2014,Huang2015a,Huang2016}. 
The research attention was oriented towards the potentials and challenges of citizens' collective actions, pro-social values and sense of community. 
The composition of the project consortium included a diversity of disciplinary profiles -- electrical engineers, computer scientists, HCI designers and sociologists -- that was necessary for tackling ST challenges in the project from multiple perspectives. % as discussed in Section~\ref{sec:design}. 
% 



%These interests were built upon prior research and ST trends regarding smart grids.
%For instance, research topics linking the potential of Social Networks (SNs) with that of smart grid applications 
%have caught great attention in recent years, following the success of several popular platforms 
%\cite{Boslet2010,Chima2011,Erickson2012,Fang2013,Huang2015}. 
%Some research conducted surveys to understand user needs for energy services combining SNs \cite{Silva2012}. 
%Some studied connecting smart meters (or smart homes) for energy management and
%sharing \cite{Ciuciu2012,Steinheimer2012}. 
%Simulation models were developed to study value-added web services \cite{De-Haan2011,Lei2012,Chatzidimitriou2013} and to 
%demonstrate the feasibility of coordination in meeting energy targets \cite{Worm2013,Skopik2014}.
%There has been works that visualize smart meter and appliance-level consumption data to enable comparative feedback among households \cite{Petkov2011,Weiss2012,Dillahunt2014}.

Another overarching goal of the CIVIS project was to integrate the core features of CIVIS design and its underlying infrastructure into rather different contexts, to meet diverse needs and expectations as well as to serve various types of users. 
This is why the pilot sites of CIVIS -- two sites hosted in Italy and two in Sweden -- were also deemed as sources of collaborative design and development rather than merely passive recipients of technologies to be tested.  

In the two Italian pilot sites\footnote{Two municipalities of Storo and San Lorenzo in Trento, Northwest Italy.}, the focus (at the community level) was cooperative owned electricity provision to local houses. 
Two electricity cooperatives, producing and 
selling 100\% renewable energy to their associate members, together with two samples of recruited 
associate member households acted as the main stakeholders.
The regional distribution system 
operator (DSO), the institutional representatives of the two municipalities, and two local cultural 
associations participated as stakeholders in different phases of the project, by providing knowledge 
and support for technical aspects related to energy and households engagement.
The CIVIS design in Italy needs to support energy communities in demand-side management\footnote{For example, moving peaks of electricity demand towards peaks of local energy
production or, in other words, improving the self-consumption capabilities of the electric cooperatives and
their associate members}. 
% 

In the two Swedish pilot sites\footnote{The neighbourhoods of Hammarby Sj\"{o}stad and F\aa{}rdala in the Stockholm area.}, the focus (at the community level) was housing cooperative's energy management in apartment buildings and town-houses.  
One site included apartment buildings owned by housing 
cooperatives\footnote{In Sweden, those who buy an apartment must join a corresponding \textit{housing cooperative} that owns and maintains the estates. The 
members of a cooperative annually elect a board that makes energy related decisions on behalf of 
the members.}. Recruited households from the cooperatives, and the cooperatives' board members acted
as key stakeholders. % Was YouPower released in Fardala at all? Do we want to explicitly talk about Fardala?
The other site was a townhouse area where the 
local residents' association and some of its member households participated to CIVIS.
The design in Sweden needs to support knowledge sharing about energy management practices at building and household levels.

%CIVIS was a three-year project that pursued the design, prototyping and real-world piloting of a platform for the improvement of energy consumption behaviour in the domestic sector. 
The project was 
structured around three main areas of interest -- energy, ICT, and social innovation 
-- and was organized in three broad phases that roughly overlapped with the three project years.
Each phase ensured a close interaction with the local realities and context of the pilot sites: (I) an 
exploratory phase, aligned and refined CIVIS' objectives with the local context,
(II) a real-world prototyping phase, concerned with the design and development of the platform 
(from data monitoring devices to the front-end applications), and (III) a piloting 
phase, for the full scale deployment of the platform in the pilot sites and 
assessment.

%Ultimately, and at the general level, the design problem areas converged on two different sets of problems depending on the two countries. 

% Does this "Evaluation" belong here? I don't think so (gp)
%\paragraph{Evaluation} After a successful deployment, YouPower was evaluated in the test sites in Sweden and Italy. We collected in parallel, the data on user engagement with the app and on energy prosumption. The initial evaluation revealed how the Trentino residents engaged with a weather feature that predicts the solar energy production levels, while the Stockholm housing association managers used the community features to engage with the residents and connect with other managers. 
%
%Some of the tangible results in the Italian test site include the savings in the electricity and heating energy spent or produced \cite{civics}:
%\begin{itemize}
%	\setlength{\itemsep}{0mm}
%	\item percent	of	self-consumption	of	 the	PV	self-produced energy	 is	 increased	 for more than	50\% of the	users	comparing	to	the period before CIVIS;
%	\item electricity	consumption	from the	grid	is	reduced for the same period for	more than	50\% of the	consumers;	
%	\item 	total	 electricity	 consumption	 (including both, from	the grid	 and	the PV	 self-consumption)		is	reduced for the same period for	more than	50\% of the	consumers;	
%	\item  the users spent less than 11\% of hours in overheating their spaces.\footnote{The	heating behaviour	 of	 CIVIS	 users (control	 of	 space	 heating),	 met	 the	recommended	standard	values.}
%\end{itemize}
%
%The housing cooperatives (BRF; bostadsr\"attsf\"orening	in	Swedish) in Stockholm conducted following actions as a result of the YouPower use \cite{civics}:
%
%\begin{itemize}
%	\setlength{\itemsep}{0mm}
%	\item BRF	Grynnan: adjustments of	ventilation	system and	turning	off	the	outdoor	ice	melting	system (resulting in	reductions	in	heating);
%	\item BRF	 Sj\"ostaden	 1: extra	 insulation	 added	 to	 the	 roofs in	May	2015	(lead	to	a	decrease	in	heating	consumption);	 installation	 of	 heat	 recovery	 heat	 pumps	 in	 February	 2016 (lead	 to	 some	 increase	 in	 electricity	 consumption	 but	 the	 overall	savings);
%	\item BRF	\"Alven: ventilation	optimization due to lower thermal comfort,	however	the	energy	use	went	up;
%	\item BRF	 Seglatsen: installation of recovery	heat	pumps	in	 the	building (reduction in	 the	heating	consumption	by	around	60\%, and since electrcitiy consumption increased, overall savings are	about	40\%).	
%	\item BRF	 Hammarby	 Kanal:	 ventilation optimization;
%	\item BRF	 Hammarby	 Ekbacke: a goal	 based	 energy	 reduction in	a	new	business	model	for	the	housing	associations	where	 they do not	 have	 to	 pay upfront	costs	for	the energy 	actions, and then part	of	the	savings	goes to	the	 ESCO (European Skills, Competences, Qualifications and Occupations)	 for	 a	 fixed	 time	 period.	
%\end{itemize}
%
%
%To assess the long term effectiveness of the CIVIS social energy intervention,  after a certain period of time, it would be beneficial to fuse the collected data from the app with those about the consumption/prosumption. 
%
%\paragraph{Limitations} The initial lack of data due to acquisition and privacy issues in the residential setting was among the main limitation in this project. We installed the IoT sensors to collect additionally needed data for using the introduced app. 

\subsection{Collaborative Design Process}
\label{sec:designProcess}
% \begin{svgraybox}
% [note by GP] Giacomo can write most of this part. 
% \end{svgraybox}

The CIVIS design process was theory-driven, human-centred, collaborative and iterative. A
literature review was carried out early in the project and later updated regarding energy intervention strategies and social smart
grid applications for the promotion of environmental behaviour change. This provided a broad set
of initial design ideas which had been iteratively assessed, expanded, refined and improved throughout the design process with the collaboration and participation of stakeholders affected.
% 
The rationale behind this approach rested on the conviction that applying a human-centred and
collaborative design process to the development of large STS has positive
theoretical, practical and ethical implications \cite{Boedker2004,Greenbaum1993} by, for instance, increasing users engagement,
usability and integration into existing local conditions \cite{Brynjarsdottir2012,Dick2012,Pierce2012}.
% 
Along the three project years, the process unfolded as a complex and articulated network of meetings and artefacts
which strived to align the interests of different stakeholders involved, from porject partners to
local stakeholders and end-users. The project team organized brainstorming sessions and design workshops, and run exploratory and evaluation focus groups with end-users in the pilot sites.
Due to limited space, the main aspects of the process are summarized as follows. Interested readers can refer to
\cite{poderi_disentangling_2017} for more detail  on how the process shaped the main
outcomes of CIVIS. % Do we want to refer to CIVIS as the project, or CIVIS platform or avoid alltogether to refer to it as an entity (once we have introduced it)


\subruninhead{User Stories} 
%\subsubsection{User stories}
We adopted the tool of user stories \cite{Kankainen2012} from Software Engineering and adapted it to the STcontext of the project. The user stories crossed CIVIS both horizontally (to the
scope of the work packages) and vertically (to the needs of the two countries). Each user
story identified a realistic scenario, a main scope of the energy intervention, the supporting ICT tools,
and the central social dynamics. During the three years, we drafted, refined, merged, abandoned and
finalized them as part of our constant work of alignment and negotiation. We discussed them in internal
workshops, round-tables with stakeholders, and focus-groups with participant end-users; we circulated
them to software engineers and platform designers; we publicly presented them for feedback and used them
as frames for collaborative workshops. They represented evolving artefacts that we consolidated in formal
versions at the end of every year of activity. 

\subruninhead{Stakeholder Meetings} 
%\subsubsection{Stakeholders meetings} % MAYBE TURN THESE INTO \PARAGRAPHS ?

These were held primarily at the level of pilot sites by involving CIVIS technical figures and the
key local energy stakeholders. Meetings were held quarterly, although at the project's onset and during
the most intense design phase, they occurred more frequently.
% 
These meetings proved helpful for agreeing on the project overarching objectives at the local levels, but also
for understanding the feasibility and rationality of the choices for the social and technical aspects of the platform.
For instance, it required long discussions and negotiation about the identification and selection of the energy monitoring devices to be installed in
participants households for enabling the proper granularity and availability of energy data. 
The suitability of these devices could not be assessed at a technical level only (regarding cost/efficiency, type of data,
reliability and protocols).
The typology of end-users and the housing conditions\footnote{In italy, participants were older and less tech-savy, living in independent, large houses; while in Sweden participants were relatively young and more tech-savy, but living in smaller apartments in residential buildings.}
also played an important role. 

%
\begin{figure}[b]
      \sidecaption[b]
        \includegraphics[width=.32\linewidth]{img/Stkh_plenary1.jpg}
	        \includegraphics[width=.32\linewidth]{img/Stkh_meeting_tou_crop.jpg} 
    \caption{(a) First project plenary meeting where local stakeholders took part; (b) Stakeholder meeting among technical project partners and
    local stakeholders to discuss demand-side management
}
\label{fig:stkh_meetings}
\end{figure}
%

%
% Another example that can be brought up relate to the availability of DSO data, which took a long time to be clarified and it also influenced/related a lot with the
% possible/desired features of the platform.
%

\subruninhead{Focus Groups} 
%\subsubsection{Focus groups} % MAYBE TURN THESE INTO \PARAGRAPHS ?

%
\begin{figure}[t]
%	\centering
	\sidecaption[t]
	\includegraphics[width=.45\linewidth]{img/FocusGroup_TN.jpg}
	\caption{An initial moment of an exploratory focus group in Italy.}
	\label{fig:focusgroups}
\end{figure}


These activities involved potential and actual participating household members, recruited for the project, and they
were run as collective discussions. Usually they lasted around two hours and included between six to eight discussants.
In case of the exploratory meetings, the scope of the discussion was intentionally broad and it aimed at revealing possible latent
needs or expectations, as well as discussing explicit ones. More importantly these were used to get first-hand knowledge
about the social and cultural environment where the platform was to be deployed. On the contrary, the evaluation discussions
had more specific focuses and involved concrete artefacts (e.g. an interface mock-up or app prototype)
as a basis.
% 
For instance, exploratory meetings helped us put in due perspective some of the features we initially taught
would be welcomed by end-users, such as ``sharing'' of energy performances or measurements typical of social network platforms.
In our contexts, it was both difficult to grasp the meaning of such a feature, but it also raised
 concerns to privacy. At the same time, the intermediate evaluation activities allowed us to spot
limitations of our data visualizations (e.g. oversimplifications of energy data through a certain type of charts),
and of the engagement and participatory process itself\footnote{A study of the end-users appreciation of the engagement and participatory process in the Italian pilot sites
is published in \cite{capaccioli_exploring_2017}.}
(e.g. expectation of more frequent interactions with the project).

\subruninhead{Design workshops} 
%\subsubsection{Design workshops} % MAYBE TURN THESE INTO \PARAGRAPHS ?

\begin{figure}[t]
      	\sidecaption[t]
        \includegraphics[width=.3\linewidth]{img/Workshop_userreq1.jpg}
	        \includegraphics[width=.3\linewidth]{img/Workshop_userreq2.jpg} 
    \caption{(a) Beginning of group activities in one of the first workshops held in Italy and focusing
    on user requirements; (b) One of the group outcomes for mapping energy consumption habits at home. 
}
\label{fig:workshops}
\end{figure}

These workshops involved concrete hands-on activities done primarily with participant household members.
Occasionally a few workshops took place among project partners or had a broader target.
As it is typical of collaborative design approaches we adopted different workshop methodologies (\textit{e.g.} brainstorming, future scenarios,
collages, usage simulation) to suit diverse needs in the different phases of CIVIS.
Ultimately, they allowed to identify the end-user requirements\footnote{A preliminary analysis of these emerging requirements in the Italian pilots
is presented in \cite{capaccioli_participatory_2016}.} for the platform front-end as well as
improving and tailoring the interface layout.
% 
For instance, for the module of \textit{Action suggestions}, the workshops were relevant for adjusting the various
tips for energy conservation to the local contexts of use. These were in fact quite different between the two countries,
 and certain tips had no meaning when delivered to one or another country or they needed a different
 rationale for their presentation.
 

\begin{sidewaystable}
\vspace*{11.5cm} 
\label{tab:activities}  
\caption{A simplified view of the relationships among type of activities the stakeholders involved in the collaborative design process and their influence on CIVIS platform design - viewed through the perspective of a socio-technical approach. Aspects that had a specific link with one of the two pilot countries (Italy ITA Sweden SWE) are reported in the table.}
\renewcommand{\arraystretch}{0.8}
\begin{tabular}{>{\centering\arraybackslash}m{5.3cm}>{\centering\arraybackslash}m{6.5cm}>{\centering\arraybackslash}m{7cm}}
\hline\noalign{\smallskip}
TYPE OF ACTIVITY\par Main stakeholders involved   & SOCIAL LEVELS & TECHNICAL LEVELS  \\
\svhline\noalign{\smallskip}
STAKEHOLDER MEETINGS \par
Project partners, institutional local energy stakeholder
&   \begin{compactitem}
	\item Endorsement and preparation of recruitment strategy for participant households
	\item Refinement and public endorsement of Participatory Energy Budgeting (ITA)
\end{compactitem} \vspace*{-.3cm} \vspace*{-.2cm}
&  \begin{compactitem}
	\item Definition of main energy targets: demand-side management (ITA), energy knowledge sharing (SWE)
	\item	Refinement of energy optimization models and feasibility of a Time-of-use signal for demand-side management
	\item 	Selection of optimal energy monitoring devices: CurrentCost (ITA), Smappee (SWE)
	\item 	Understanding of DSO energy data structure and availability
	\item 	Understanding of existing energy/ICT infrastructure
	\item 	Availability to invest ``energy bonus'' for Participatory Energy Budgeting  (ITA)
\end{compactitem} \vspace*{-.3cm} \vspace*{-.2cm} \\
  \hline\noalign{\smallskip}
FOCUS GROUPS \par
Recruited household members 
&  \savespace \begin{compactitem}
 	\item Understanding local context: strength of local community groups and associations reinforcing the idea to promote joint actions through the platform 
 	\item Emerging concerns about privacy
 	\item Emerging concerns about anonymity related to energy data comparisons (ITA)
 	\item Exploration of ``ICT literacy''
 \end{compactitem} \vspace*{-.3cm} \vspace*{-.2cm}
  &  \savespace \begin{compactitem}
   	\item Exploration and rejection of social networking features
	\item Refinement of the understanding of ICT devices availability, type and use
   \end{compactitem}  \vspace*{-.3cm} \vspace*{-.2cm}  \\  \hline\noalign{\smallskip}
CO-DESIGN WORKSHOPS \par
Recruited household members, institutional local energy stakeholder, project partners \vspace*{-.3cm}
&   \savespace  \begin{compactitem}
   	\item Requested interface features: real time and historical data for PV production (ITA)
   	\item Definition of PEB policy documents (ITA)
   	\item Collaborative content-generation sessions for Housing Cooperative module (SWE)
   \end{compactitem} \vspace*{-.3cm} \vspace*{-.2cm}
   &  \savespace \begin{compactitem}
      	\item Exploration and rejection of social networking features
   	\item Refinement of the understanding of ICT devices availability, type and use
      \end{compactitem}  \vspace*{-.3cm} \vspace*{-.2cm}
      \\ \cline{2-3}\noalign{\smallskip}
& \multicolumn{2}{p{13cm}}{ \savespace \begin{compactitem}
      	\item Assessment and usability feedback on all module mock-ups, leading to improvement of interface designs
      \end{compactitem} } \vspace*{-.2cm} \vspace*{-.1cm} \\ \hline
\end{tabular}
\end{sidewaystable}



In general, a constant work of alignment took place at a high level of abstraction mainly thanks to
the use of user stories as key boundary object among stakeholders,
expertises and local contexts.
At a more concrete level, a set of platform features was prototyped in simple mock-ups
and also used as a basis for discussion. These underwent iterative rapid prototyping which
produced wireframes as better visual guides that could be more effectively communicated to end-users.
Prior and after each iteration, exploratory activities on how to proceed and evaluation sessions for their outcomes took place in different venues and with different stakeholders.
Table~\ref{tab:activities}   
provide a brief overview on the relationships among the various activities of the collaborative design process and their influence on CIVIS platform design viewed through the perspective of an ST approach. 

\subsection{Main Outcomes of the Design Process}

The main outcomes of the CIVIS collaborative design process include (1) an open source social smart grid application called YouPower \cite{Huang2017}, and (2) community engagement approaches  that were implemented during the change process of the project \cite{capaccioli_exploring_2017,Hasselqvist2015}, both contextualized to the local situations. 

\subsubsection{YouPower: An Open Source Social Smart Grid Application} %or similar heading




Combining smart sensing and web technologies among others,
YouPower is designed as a social smart grid application (developed by the CIVIS project as a hybrid mobile app) that can connect users to friends, families and local communities to learn and take energy actions that are relevant to them together. The app encourages an energy-friendly lifestyle and can be linked to users' energy consumption and production data for quasi real-time and historical prosumption information. 
% 
%\noindent Given time and resource constraints, the YouPower app can not be developed all-in-one cross-platform (for phones, tablets and computers). We chose to design the front-end as a hybrid mobile phone app, i.e. its UI design has layouts that suit phone screens, %The consideration is multi-fold. 
%Western Europe has a large mobile phone internet user base\footnote{Between 2013 and 2017, the penetration rate of mobile phone internet users among mobile phone users will rise from 49.0\% to 77.8\%. See more at: \url{ http://www.emarketer.com/Article/Nearly-Half-of-Western-Europeans-Will-Use-Mobile-Web-This-Year/1010510\#sthash.AaVfsqIU.dpuf}}. Many surveys show that mobile apps have advantages such as creating deeper user engagement, easy sharing, among others\footnote{\url{https://infomedia.com/blog/the-advantages-of-mobile-apps/}, \url{https://econsultancy.com/blog/62326-85-of-consumers-favour-apps-over-mobile-websites/}}. This makes mobile app a good choice given the goal of the CIVIS platform. 
%since mobile apps can be more easily transformed to web browser versions, while the reverse is more difficult.
%The back-end of the YouPower platform will remain mostly the same independent of the front-end alternatives.
%
The platform as a whole (shown in Figure~\ref{fig:platform}) is mainly composed of (I) the \textit{energy sensor level services} mainly
dealing with energy data collection, and (II) the \textit{energy data level and social
level services} mainly dealing with energy data analytics as well as user, household and community management
among others. 

\begin{figure}[h!]
\sidecaption[t]
%\footnotesize
	\includegraphics[width=.64\linewidth]{img/civis_platform_overview.pdf} %\\
	%DSO (Distribution System Operators),  SSL (Secure Sockets Layer)
	\caption{The CIVIS project platform overview. DSO (Distribution System Operators); SSL (Secure Sockets Layer)}\label{fig:platform}
\end{figure}


\runinhead{Energy Sensor Level Services} The CIVIS project installed hardware (smart plugs and sensors) and
software required for appliance-level energy data collection. The hardware/software choices differ in the
two sites due to the local context. For example, \textit{Smappee}\footnote{\url{http://www.smappee.com}}
for 40 households in Stockholm, and \textit{CurrentCost}\footnote{\url{http://currentcost.com}} for 79
households in Trento. Trento also installed Amperometric clamps for PV prodcution measures. 
Household-level energy data of the pilot sites in both countries is measured by smart meters and provided by local DSOs. 

\runinhead{Energy Data Level and Social Level Services} These services are provided by the YouPower
app and its back-end. The design consists of three self-contained
composable parts: (1) \textit{House Cooperatives} (contextualized and deployed to the Stockholm pilot site);
(2) \textit{Demand-Side Management} (contextualized and deployed
to the Trento pilot site); and (3) \textit{Action Suggestions} (contextualized and deployed to both pilot sites).
They are discussed in the following subsections. 

\paragraph{Housing Cooperatives}
\label{sect:brf}

This part of the YouPower app is designed for the community of housing cooperatives\footnote{\textit{Bostadsr{\"a}ttsf{\"o}rening} or \textit{Brf} in Swedish.} in the Stockholm pilot sites \cite{Hasselqvist2016}.
Similar housing ownership and management models exist in a number of EU and non-EU countries, which allow potential wider application of the design.
A housing cooperative annually elects a board which manages cooperative properties and decides on energy contracts, maintains energy systems, and proposes investments in energy efficient technologies. Since board members are volunteers who may have limited knowledge of energy or building management, this module aims to support board members in energy management, in particular energy reduction actions. Cooperative members can also use the app to follow energy decisions and works of the cooperative. Additionally, the app can be of interest by building management companies working with housing cooperatives. 
The information presented in the app is visible for these user
groups and shared between housing cooperatives. This openness of energy data is key to
facilitating  users in sharing experiences relevant for taking energy reduction actions.

% \subsubsection{Linking energy data to energy reduction actions}
\subparagraph{Linking Energy Data to Energy Reduction Actions}

The design links energy data with energy reduction actions taken (Figure~\ref{fig:Figure201_Actions}) at cooperative levels, making the impact of energy actions visible to users. The energy use is divided into distrct heating \& hot water, and facilities electricity in apartment buildings. Users can switch between the views per month or per year to show overall changes. %Since the energy data is shared between cooperatives there may also be privacy concerns related to opening up data of higher granularity to people outside of the own cooperative. 
%
\begin{figure}[t!]
	%\centering
	\sidecaption[t]
	\includegraphics[width=.64\linewidth]{img/Figure201_Actions.png}
	\caption{Heasting \& hot water use graph. Blue bars show the current year's use per month; the black line shows that of previous year. Energy reduction actions taken are mapped to the time of action and listed below.}
	\label{fig:Figure201_Actions}
\end{figure}
%
Users with editing rights, typically board members, can  add energy reduction actions that the cooperative has taken, e.g., improvement of ventilation, lighting or heating systems, 
and the related cost.
Trusted energy or building management companies can also get editing rights to add energy reduction actions they took on behalf of the cooperative. 
Added actions appear at the month when each action was taken and are listed below the graph. When clicking on an action in the list, the details of the action are shown.
% 
To make the impact of actions visible, users can compare the energy use of the viewed months to that of a previous year. This can be used e.g. by a cooperative to explore what energy reduction actions to take in the future by learning actions taken by other cooperatives and what the effects were in relation to costs.

% \subsubsection{Comparing housing cooperatives}
\subparagraph{Comparing Housing Cooperatives}

\begin{figure}[t!]
	%\centering
		\sidecaption[t]
	\includegraphics[width=0.64\linewidth]{img/Figure202_Housing_cooperatives_comparison.png}
	\caption{Map and list view of participating housing cooperatives. The energy performance of cooperatives is indicated by colour and in numbers.}
	\label{fig:Figure202_Housing_cooperatives_comparison}
\end{figure}

The cooperatives that are registered for the app are displayed in a map or list view (Figure~\ref{fig:Figure202_Housing_cooperatives_comparison}). Their icons are color coded (from red to green) based on each cooperative's energy performance, i.e. from high to low energy use per heated area, scaled according to the Swedish energy declaration for buildings\footnote{\url{http://www.boverket.se/sv/byggande/energideklaration/energideklarationens-innehall-och-sammanfattning/sammanfattningen-med-energiklasser/energiklasser-fran-ag/}}. 
%  but it is calibrated to only include measured energy use for heating and hot water, which is the greatest part of the energy use. In the Swedish energy declarations, facilities electricity is also added but that often requires estimations of different factors to make the number comparable.
% 
Users can also see the energy performance as a number (in kWh/m$^2$), and the information about energy reduction actions of the cooperatives. %The number of actions is important to display to make energy reduction efforts of housing cooperatives with a high energy performance (e.g. due to poor construction of the building) visible. 
% 
During stakeholder studies, energy managers in cooperative boards stressed the importance of knowing the difference between cooperatives in order to understand the difference in their energy performance. Thus, the design also includes information about cooperatives (Figure~\ref{fig:Figure204_Neighbourhood_average}) such as the number of apartments and heated areas in a cooperative, a building's construction year, and types of ventilations (e.g. with or without heat recovery).
% 
Users can compare a cooperative's energy use per month or per year to another cooperative or to the neighborhood average. The electricity use is also displayed per area (kWh/m$^2$) to make it comparable.
\begin{figure}[t]
	\sidecaption[t]
\includegraphics[width=.35\linewidth]{img/brf.pdf}
\caption{Facilities electricity use graph. Information about housing cooperatives and actions is displayed at the top. Green bars show the housing cooperative's current year's use per month; the black line shows the average use of all housing cooperatives}
\label{fig:Figure204_Neighbourhood_average} 
\end{figure}

% \subsubsection{Sharing experiences}
\subparagraph{Sharing Experiences}


A cooperative interested in taking an action may wish to know more, e.g. which contractor was chosen for an investment and why or how to get buy-in from cooperative members. The design provides commenting functions for each action added, where users can post questions and exchange experiences. The cooperatives can also add email addresses of their contact persons, which are visible on each cooperative's app page.
% 
Sharing experiences certainly also happens outside of the digital world, e.g. during meetings of cooperative boards or with local energy networks. The app aims  to support discussions and knowledge exchange also in such situations, where someone can easily demonstrate the impact of an energy investment with smart phones.

% % % ITALIAN pilot site
\paragraph{Demand-Side Management} 
% \label{sect:load_shifting}

This part of the YouPower app is designed for the Trento pilot site and can have wider application.
It provides users historical and quasi real-time consumption and production information, and facilitates users to leverage load elasticity in order to maximize self-consumption of rooftop PV productions. 
Energy data is displayed at appliances (if smart plugs are installed), household, and electricity consortia levels. %to inform users of their own energy consumption patterns and those in the neighborhood. 
%
Consumption at the appliance level enables users to gain deeper understanding of their daily actions and the resulting energy use. 
% 
Historical and current consumption and production at the household level allow users to compare those two and potentially maximize self-consumption. 
% 
Aggregated and average consumption at the consortia level informs users of neighborhood energy consumption and allows comparisons.  
% 
In addition, dynamic Time-of-Use (ToU) signals are displayed  to assist users in load shifting during their daily actions.

% \subsubsection{Historical and quasi real-time consumption and production} 
\subparagraph{Historical and Quasi Real-time Consumption and Production}

At the household level, electricity consumption and PV production levels (in W and Wh) are displayed in quasi real-time and updated for the latest six minutes\footnote{For technical reasons such as households' data transfer connections and processing time, there can be up to 2-min delay between the time of actual power measurement and the data displayed.}.
This information can also be displayed as a bar chart for a chosen period (in the past) to provide an aggregated daily overview of consumption vs. production (Figure~\ref{fig:viz_rt}). 
% 
\begin{figure}[b]
\sidecaption[t]
        \includegraphics[width=.25\linewidth]{img/visual_production.png}	        \includegraphics[width=.35\linewidth]{img/historicalcomparison_prodcons.png} 
    \caption{(a) Quasi real-time meters for household PV production; (b) Household consumption vs. production for a chosen period}
\label{fig:viz_rt}
\end{figure}
%
When smart plugs are installed, users can view the daily electricity consumption (in Wh) of the corresponding connected appliances of their own household for a chosen period (Figure~\ref{fig:viz_hist} a). This helps them to gain better insights into the individual appliance's consumption level and its daily or seasonal patterns. 
% Selection of data ranges are mandatory for these visualizations. They must be set by users at two different places: in the ``Energy Data'' main screen for the \textit{Household} category; in the ``light-bulb'' sub-view for the \textit{Appliance} one, which is accessible from the top level bar.
\begin{figure}[t]
      \sidecaption[t]
        \includegraphics[width=.35\linewidth]{img/applianceconsumption.png}
         \includegraphics[width=.25\linewidth]{img/benchmark.png}
      \caption{(a) Daily electricity consumption at the appliance level for a chosen period;  (b) 
      A household's hourly consumption profile over a chosen day compared to the averages and totals of the consortia}

\label{fig:viz_hist}
\end{figure}
 %
With the aggregated energy data provided by the two local electricity consortia, users can also  compare their own households' hourly consumption profiles over a chosen day to the averages and totals of the consortia to gain a sense of their relative performance compared to their peers (Figure~\ref{fig:viz_hist} b).

\begin{figure}[b]
      \sidecaption[t]
        \includegraphics[width=.3\linewidth]{img/touprediction.png}
         \includegraphics[width=.3\linewidth]{img/touperformancechart_indivcoll.png}
      \caption{(a) Dynacmie ToU signals at 3-hour intervals for the forthcoming 30 hours;  (b) 
      A household's hourly consumption profile over a chosen day compared to the averages and totals of the consortia
}
\label{fig:tou}
\end{figure}

% \subsubsection{Dynamic ToU signals} 
\subparagraph{Dynamic ToU Signals}

Dynamic ToU signals are provided to facilitate users' self-consumption of local PV productions.
They give clear indications to encourage or discourage electricity consumption at a certain moment based on the forecasted local renewable production level calculated with open weather forecast information (in particular solar radiation data) and the local rooftop PV production capacity. 
The signals are at 3-hour intervals for the forthcoming 30 hours (Figure \ref{fig:tou} a), and are updated every 24 hours. A green smiley face signals a time slot suitable for self-consumption where the forecasted local PV production exceeds the current local consumption, while an orange frown face signals otherwise.  
% 
On a weekly basis, users get a summary of the proportion of their own household consumption that took place under green or orange ToU signals to allow them to reflect on their levels of self-consumption (Figure \ref{fig:tou} b). The same information is also provided at the consortia level to enable peer comparison. 

\paragraph{Action Suggestions}
% \label{sect:tips}

This part of the YouPower app aims to %provide actionable suggestions to 
facilitate all household members to take part in energy conservation in their busy daily life. 
% 
About fifty action suggestions are composed to provide users practical and accurate information about energy conservation. 
They include one-time actions such as ``Use energy efficient cooktops'', routine actions such as ``Line dry, air dry clothes whenever you can'', as well as in-between actions (reminders) such as ``Defrost your fridge regularly (in $x$ days)''. 
Some suggestions may seem obvious and trivial, but as indicated by literature, people often has an attitude-behavior gap when it comes to environmental issues. The goal is to facilitate the behavior change process to bridge the attitude-behavior gap, making energy conservation new habits integrated in everyday household practices. 

% \subsubsection{Free choice and self-monitoring of energy conservation actions}
\subparagraph{Free Choice and Self-monitoring of Energy Conservation Actions}

The actions are not meant as prescriptions for what users should do but to present different ideas of what they can do (and how) in household practices. 
Users can freely choose whether (and when) to take an action and possibly reschedule and repeat the action according to the needs and interests in their own context (Figure \ref{fig:actions}). After all, users are experts of their own reality. They also have an overview of their current, pending, and completed actions.
A new action is suggested when one is completed. %After an action is in progress, the user may also postpone, abandon or indicate that the action is completed (Figure \ref{fig:actions} c). 
When an action is scheduled, its reminder is triggered by time. Users' own choices of actions and the action processes facilitate the sense of autonomy which enhances and maintains motivation \cite{Ryan2000}.

\begin{figure}[b!]
      \begin{center}
        \begin{minipage}[t!]{0.33\linewidth}
	       \includegraphics[width=1\linewidth]{img/action_details.jpg}
        \end{minipage}
        \begin{minipage}[t!]{0.31\linewidth}
        	       \includegraphics[width=1\linewidth]{img/Your_Actions.jpg}
                \end{minipage}
        %\hfill 
        \begin{minipage}[t!]{0.33\linewidth}    
         \includegraphics[width=1\linewidth]{img/action_tab.jpg}    
        \end{minipage}
      \end{center}
      \caption{(a) Action suggestion; (b) Action in progress; (c) User actions}\label{fig:actions}
\end{figure}


% \subsubsection{Promoting motivation and engagement} 
\subparagraph{Promoting Motivation and Engagement}

The design uses a number of elements to promote users' motivation and engagement. 
The suggestions are tailored to the local context by local partners and focus groups. 
Each action is accompanied by a short explanation, the entailed effort and impact (on a five-point scale) and the number of users taking this action. 
The design encourages users to take small steps (and not to have too many actions at a time) and gives positive performance feedback. 
In addition, users can invite household members, %  (Figure \ref{fig:invite}), 
view and join the energy conservation actions of the whole household. % (Figure \ref{fig:form} a).
Users can also login with Facebook, like, comment, share actions, % (Figure \ref{fig:share}), 
give feedback %(Figure \ref{fig:form} b c) 
and invite friends. Users are awarded with points  (displayed as Green Leaves) once they complete an action, or provide feedback or comments. 


%\begin{figure}
%      \begin{center}
%      \begin{minipage}[t!]{0.33\linewidth}    
%               \includegraphics[width=1\linewidth]{img/house2.jpg}    
%       \end{minipage}
%        \begin{minipage}[t!]{0.33\linewidth}
%	       \includegraphics[width=1\linewidth]{img/action_not_completed.pdf}
%        \end{minipage}
%        \begin{minipage}[t!]{0.31\linewidth}
%        	       \includegraphics[width=1\linewidth]{img/action_completed.pdf}
%                \end{minipage}
%        %\hfill 
%      \end{center}
%      \caption{(a) Household actions; (b) Feedback form -- action abandoned; (c) Feedback form -- action completed}\label{fig:form}
%\end{figure}



\subsubsection{Community Engagement Approaches} %or another nicer subsec-heading
% \begin{svgraybox}
% [note by GP] Participatory energy budgeting process in trento + Training sessions 
% in sweden (if you find any info about these Swe activities - from deliverables or kth publications, 
% please send me a note).
% \end{svgraybox}

Another main outcome of the design process, which also reflects the potential richness of designing for large-scale STS, rests at the level of community engagement. %The approaches to the use, adoption and appropriation of the platform resulted for the pilots in the two countries. % I don't understand this sentence
The ambition to foster energy behaviour change at the collective level of communities (or neighbourhoods),
instead of simply aiming for technology adoption at individual level, made it clear the need to design for
engagement.
% 
In the two national contexts, two different engagement processes accompanied the deployment and testing of the technology. They tried to stimulate the emergence of the social dynamics connected to the change of energy behaviour. (Note that the collaborative design process discussed in Subsection~\ref{sec:designProcess} also contributed to engagement.)

In Italy, a full fledged process named \textit{Participatory Energy Budgeting} (PEB) \cite{capaccioli_exploring_2017,capaccioli_exploring_2016}
was run with the twofold goal of subsidizing people's efforts in demand-side management
and empowering them to handle their achievements in a collective and transparent way.
%
PEB is a policy frame that relies on a call for tender that
defines: the energy budget to be administered;
the criteria and procedures to submit proposals for funds request; the procedures
to evaluate, select and award the winning proposals; and a roadmap for the process development. 
Grounded on the community funds model of participatory budgeting \cite{Ganuza2012,Sintomer2008},
PEB promoted engagement and allowed collective decision
making around the management and allocation of ``energy bonus'', which could be
collected through the collective effort of shifting electric energy demand
towards local production peaks.
% 
The PEB and the demand-side management module of YouPower
were thought and designed to act in synergy and ``reinforce'' each other.
%Such bonuses are linked to the community performances related to demand-side management:
The more people consumed energy during peaks of local production -- foretold and displayed
with ``green smileys'' in YouPower -- the more the energy bonuses grew.
% 
PEB can be considered a main outcome of the design process in the Italian sites, because 
the idea to manage collectively the energy savings emerged during the first exploratory focus groups,
and throughout the fist two project years, such idea has been refined and negotiated into
a full-fledged policy frame, with the participation of recruited households and the endorsement
of the electric consortia. For instance, while the latter vouched for the legitimacy of the process
and made the ``energy bonus'' practically available, the former defined key aspects of PEB frame such as the criteria for eligibility and those ones for final evaluation and award.  


% TEXT BY HANNA H
In Sweden, the engagement work and app design aimed to complement the already existing community efforts to address energy issues. Meetings were arranged with housing cooperative representatives to discuss experiences of energy reduction actions and how those could be shared through the Youpower app. Furthermore, the app was used as a probe to discuss housing cooperative energy management with other stakeholders who may influence housing cooperative energy use, such as building managers, energy providers, and energy advisers. These stakeholders were already working with housing cooperatives and many had ambitions of supporting housing cooperatives in reducing energy use. By engaging with these stakeholders and learning about their processes and goals, we identified opportunities for the Youpower app to be used jointly by these stakeholders and housing cooperatives to support energy improvement work.

