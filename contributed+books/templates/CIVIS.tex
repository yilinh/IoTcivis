\section{CIVIS: A Community-Oriented Design in Future Smart Grids}

\begin{svgraybox}
Discuss the CIVIS project making relation to the previous theory section. 

The discussion shall not be limited to the app YouPower, but also the other efforts made around it (if they are related to the discussions in the previous section), e.g. the user stories, focus groups workshops, interviews, participatory budgeting, etc. 

Use the YouPower paper as much as possible. 
\end{svgraybox}

CIVIS was a three year, EU-funded project\footnote{\url{http://cordis.europa.eu/project/rcn/110429\_en.html}} that pursued the design, prototyping and real-life testing of a platform
for the improvement of energy behaviours in the domestic sector. The project was structured around three main areas
of interest – \textit{i.e.} energy, ICT, social innovation and business – and organized into three broad phases that roughly
overlapped with the project years: (\textit{i}) an exploratory one, used to align CIVIS’ overarching objectives with the local contexts’ needs;
(\textit{ii}) a prototyping one, which concerned the actual design and development of the platform (from data monitoring devices to the
front-end applications); and (\textit{iii}) a final testing phase which included the full scale deployment of the platform in the pilots for usage and assessment purposes.

\subsection{Understanding and Formulation of the Design Situation}
\begin{svgraybox}
[note by GP] Giacomo can write this part. To be expanded/integrated by all.

It will focus on CIVIS project by outlining its foundations (as fp7 R\&I project) and approach (interaction with local stakeholders + incremental design \& development of platform architecture). It will describe the inital, explicit objectives (+ simple rationale/roadmapp to achieve them) and it will provide an overview of the local contexts (from social, technical and energy point of view), in order to make clearer who are the involved stakeholders and their context.
\end{svgraybox}

\subsection{Products of the Design Process}

\subsection{Design Process (or participatory design?)}

\begin{svgraybox}
[note by GP] Giacomo can write this part. 

\end{svgraybox}


\section{Discussions}

\begin{svgraybox}
Lessons Learned?  / Design Guidelines? 
\end{svgraybox}