\section{CIVIS: A Community-Oriented Design in Future Smart Grids}

\begin{svgraybox}
Discuss the CIVIS project making relation to the previous theory section. 

The discussion shall not be limited to the app YouPower, but also the other efforts made around it (if they are related to the discussions in the previous section), e.g. the user stories, focus groups workshops, interviews, participatory budgeting, etc. 

Use the YouPower paper as much as possible. 
\end{svgraybox}

CIVIS was a three year, EU project\footnote{\url{http://cordis.europa.eu/project/rcn/110429\_en.html}} funded under the \textit{FP7 Smart Cities} framework, that pursued the design,
prototyping and real-life testing of a platform for the improvement of energy behaviours in the domestic sector. The project was structured around three main areas
of interest – \textit{i.e.} energy, ICT, social innovation and business – and organized into three broad phases that roughly
overlapped with the project years and that ensured a close interaction with the local realities and contexts of
the pilots: (\textit{i}) an exploratory one, used to align CIVIS’ overarching objectives with the local contexts’ needs;
(\textit{ii}) a prototyping one, which concerned the actual design and development of the platform (from data monitoring devices to the
front-end applications); and (\textit{iii}) a final testing phase which included the full scale deployment of the platform in the pilots for usage and assessment purposes.

Italy and Sweden hosted two pilots each. 
The two municipality areas of Storo and San Lorenzo, in Trentino Alto-Adige (a region in north-west Italy), included the
Italian pilots. Here, two electric cooperatives, producing and selling 100\% renewable energy
to their associate members, together with two samples of recruited participant households acted as the main stakeholders.
The regional distribution system operator (DSO),
the institutional representatives of the two municipalities and two local cultural associations participated as
stakeholders in various phases of the project, by providing knowledge and support for technical aspects related
to energy and households engagement.
\textbf{TO BE FIXED:} The area of Stockholm hosted the two Swedish pilots. One involved the residential and central neighbourhood
of Hammarby Sj\"{o}stad, which mainly includes apartment buildings owned by housing cooperatives. Recruited households from the cooperatives and cooperatives' board members acted
as key stakeholders here. The other pilot concerned a townhouse area in the outskirts of Stockholm: F\aa{}rdala. In this townhouse
area the local residents’ association and some of its member households participated to CIVIS.

\subsection{Understanding and Formulation of the Design Situation}
\begin{svgraybox}
[note by GP] Giacomo can write this part. To be expanded/integrated by all.

It will focus on CIVIS project by outlining its foundations (as fp7 R\&I project) and approach (interaction with local stakeholders + incremental design \& development of platform architecture). It will describe the inital, explicit objectives (+ simple rationale/roadmapp to achieve them) and it will provide an overview of the local contexts (from social, technical and energy point of view), in order to make clearer who are the involved stakeholders and their context.
\end{svgraybox}

Since more than two decades the ongoing and long-term energy transition process is shifting the energy domain
towards decentralization, distributed production and renewable sources \cite{rifkin_third_2011; sovacool_how_2016}.
Several general and intertwined aspects contribute to this transition: (\textit{i}) the awareness of the inherent complexities
that exist among energy systems, societies and the environment \cite{bulkeley_bringing_2012; umbach_global_2010}; (\textit{ii}) the
widespread diffusion of new, enhanced technologies and their hybridization with contemporary ICTs \cite{putrus_smart_2013; schick_innovating_2013};
(\textit{iii}) the pursuit of national and supranational energy policies around energy efficiency, sustainability and low carbon emissions \cite{da_graca_carvalho_eu_2012};
and (\textit{iv}) the emergence of new actors in the energy value chain, such as energy cooperatives and
energy communities \cite{viardot_role_2013}, or the transformation of old ones, such as housing associations,
and amateur energy managers \cite{hasselqvist_linking_2016}.



\subsection{Products of the Design Process}

\subsection{Design Process (or participatory design?)}

\begin{svgraybox}
[note by GP] Giacomo can write this part. 

\end{svgraybox}


\section{Discussions}

\begin{svgraybox}
Lessons Learned?  / Design Guidelines? 
\end{svgraybox}