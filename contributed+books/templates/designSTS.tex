\section{Designing in Large-scale Socio-technical Systems}
\label{sec:design}
% Always give a unique label
% and use \ref{<label>} for cross-references
% and \cite{<label>} for bibliographic references
% use \sectionmark{}
% to alter or adjust the section heading in the running head

\begin{svgraybox}
The socio-technical view can be articulated as the recognition of (1) the mutual constitution of people and technologies, (2) the contextual embeddedness of this mutuality, and (3) the importance of collective action \cite{Sawyer2014}. 
Those who hold this view examine more than just the technological system, or just the social system, or even the two side by side, but also the phenomena that emerge when the two interact \cite{Lee2001}. 
\end{svgraybox}



\begin{svgraybox}

Designing large-scale complex systems with a socio-technical view has a number of implications for (1) the formulation of the design problem, (2) the products of the design process, and (3) the design process itself (BootCamp, BC).

\runinhead{Formulation of the design problem} It is not straightforward what needs to be taken into consideration in relation to the design. What systems boundaries to choose. the question of systems boundaries is an issue for technical systems and even becomes more difficult for social technical systems
what are the issues to be addressed.[BC]

Ill-structured problem

\runinhead{Products of the design process} these not only consists of technological artifacts but also may include rules for behaviour, policies, etc. through which the designer wish to intervene in social-technical systems. what is it that we are designing? 

\runinhead{Design process} The design process can be seen as a decision-making process where the problem owners, shareholders, users, etc. participate to represent their interests. It is often conceived and implemented in participatory decision-making processes actively involving stakeholders

Large-scale socio-technical systems are often not designed as a whole but incrementally ``piece by piece'' evolving from legacy systems (BC). Designers are therefore working \textit{in} the context of some socio-technical system with the intention of changing or improving some part of that system [BC]. This means that what matters more in the design is the design process itself, more than the ``final status'' of the system \cite{Shin2014, need more ref} because the socio-technical system keeps evolving and exhibits emergent behaviour \cite{Nikolic2009}. An important  goal of the design process is to make the design (a product or system) relevant to the evolving context \cite{Shin2014, need more ref} as social and technical artifacts exist within their socio-technical context [BC]. 

\end{svgraybox}






\begin{svgraybox}
Use and combine content in:
\begin{enumerate}
\item \cite{Norman2015} (design problems in large-scale socio-technical systems) and 
\item \cite{Baxter2011} (socio-technical approach to systems engineering)
\item \cite{Whitworth2009} (four system levels of Socio-tehnical systems); 
\item \cite{Shin2014} (a very good article about IoT, socio-technical perspective )
\item see also https://medium.com/rettigs-notes/notes-on-sociotechnical-systems-design-178f161bc9e8 
\end{enumerate}
\end{svgraybox}

