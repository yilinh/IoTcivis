\section{Designing in Large-scale Socio-technical Systems}
\label{sec:design}
% Always give a unique label
% and use \ref{<label>} for cross-references
% and \cite{<label>} for bibliographic references
% use \sectionmark{}
% to alter or adjust the section heading in the running head

The term ``socio-technical'' embodies both a research perspective and a subject matter \cite{Lee2001}.
This can be articulated as the recognition of three properties \cite{Sawyer2014} as follows. 
%
First, the \textit{mutual constitution of people and technologies}. This mutual constitution generates complex and dynamic interactions among technological capacities, social norms, histories, situated context, human choices and actions, etc. 
%
Second, the \textit{contextual embeddedness of the mutuality}, where the context is not taken as fixed or delineable. There are dynamic situational and temporal conditions that influence 
mutual adaptations throughout the course of design, development, deployment and uses of the system of interest. 
% 
Third, the \textit{importance of collective action}, the joint pursuit of one or more shared (potentially conflicting) goals by two or more interested parties (without implying positive or negative outcomes). The collective action shapes and is shaped by both the context and the technological components. 

Researchers who hold a socio-technical systems view investigate more than just the technological system or just the social system or even the two side by side, but also the phenomena that emerge when the two interact \cite{Lee2001}. A socio-technical approach tries to abstain from oversimplifications that seek a single or dominant cause of change, but studies the complexity, dynamic and uncertainty in the networks of institution, people and technological artifacts in the process of technologically involved change \cite{Sawyer2014}. 
%
Taking a socio-technical approach to design has a number of implications for (i) the formulation of the design problem, (ii) the products of the design process, and (iii) the design process itself (BootCamp, BC?).

\begin{svgraybox}
\runinhead{Formulation of the design problem}

The understanding and formulation of the design problem 

 It is not straightforward what needs to be taken into consideration in relation to the design. What systems boundaries to choose. the question of systems boundaries is an issue for technical systems and even becomes more difficult for social technical systems
what are the issues to be addressed.[BC]

Ill-structured problem

\runinhead{Products of the design process} these not only consists of technological artifacts but also may include rules for behaviour, policies, etc. through which the designer wish to intervene in social-technical systems. what is it that we are designing? 

\runinhead{Design process} The design process can be seen as a decision-making process where the problem owners, shareholders, users, etc. participate to represent their interests. It is often conceived and implemented in participatory decision-making processes actively involving stakeholders

Large-scale socio-technical systems are often not designed as a whole but incrementally ``piece by piece'' evolving from legacy systems (BC). Designers are therefore working \textit{in} the context of some socio-technical system with the intention of changing or improving some part of that system [BC]. This means that what matters more in the design is the design process itself, more than the ``final status'' of the system \cite{Shin2014, need more ref} because the socio-technical system keeps evolving and exhibits emergent behaviour \cite{Nikolic2009}. An important  goal of the design process is to make the design (a product or system) relevant to the evolving context \cite{Shin2014, need more ref} as social and technical artifacts exist within their socio-technical context [BC]. 

\end{svgraybox}



\begin{svgraybox}
a-contextual and detemporalized perspective approaches , general solution ., is self-limiting 
focus on situating work and seek to examine all contextual factors , this types of inquiry attempt to construct a holistic view of context: one that does not diminish or remove contextual elements, even those with limited influence. 
paying little attention to the environment of the organization and temporal dimension of technological innovation 
\end{svgraybox}


\begin{svgraybox}
Use and combine content in:
\begin{enumerate}
\item \cite{Norman2015} (design problems in large-scale socio-technical systems) and 
\item \cite{Baxter2011} (socio-technical approach to systems engineering)
\item \cite{Whitworth2009} (four system levels of Socio-tehnical systems); 
\item \cite{Shin2014} (a very good article about IoT, socio-technical perspective )
\item see also https://medium.com/rettigs-notes/notes-on-sociotechnical-systems-design-178f161bc9e8 
\end{enumerate}
\end{svgraybox}

