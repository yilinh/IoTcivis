\section{IoT for the Smart Grid}
\label{sec:IoTSG}
The term Internet of Things (IoT) was coined by Ashton in 1999 \cite{ashton2011internet} while introducing RFID technology in the context of supply-chain management. The meaning has evolved during the past years. International Telecommunication Union (lTU) defines IoT as the worldwide network of interconnected objects uniquely addressable based on standard communication protocols. While the Internet has led to the interconnection of people at an unprecedented scale, the IoT is expected to interconnect also the objects around us, leading to a smart environment \cite{gubbi2013internet}.	

On the most general level the IoT can be divided into three following layers:
\begin{itemize}
\item comprehensive sensing (perception layer),
\item reliable transmission (network layer), and
\item intelligent processing (application layer).
\end{itemize}
	
Another term that is often used interchangeably with IoT is \textit{ubiquitous computing} coined by Weiser \cite{weiser1991computer}. Ubiquitous computing is defined as ``the physical world that is richly and invisibly interwoven with sensors, actuators, displays, and computational elements, embedded seamlessly in the everyday objects of our lives, and connected through a continuous network'' (ibid.). While the IoT describes connected devices, ubiquitous computing focuses on the smart environment in which computing is pervasive. Hence, the two terms take a different stance and focus on different aspects of what is envisioned to become the Future Internet. 
The vision put forward by Weiser \cite{weiser1991computer} led to a fruitful new field within computer science (ubiquitous computing). However, 15 years later, Rogers \cite{rogers2006moving} offered a constructive critique of this vision. Namely, Rogers argued that we should switch from a computing approach to a human approach in developing the smart environment. In particular, the original vision suggested that ubiquitous computing can lead to an environment that is predicting and adapting to the people's needs, while the people were considered passive elements. Rogers argues the opposite: ``To make this happen, however, requires moving from a mindset that wants to make the  environment smart and proactive to one that enables people, themselves, to be smarter and proactive in their everyday and working practices.''

Yun and Yuxin \cite{yun2010research} discuss the possibilities of the IoT to bring about the \textit{smart grid} through sensors, novel telecommunications and computing technologies. The sensors, such as smart, temperature, and illumination meters, collect energy and environmental data. They can also form a high-speed, real-time and bidirectional connection between the consumers, utilities and the electrical grid. It is envisioned that such an improved data collection and communication can support the decision making and in turn improve the overall efficiency of the grid. Interestingly, the technology at the heart of the IoT, the Internet itself, consumes up to 5\% of the total energy spent today in the world. Given the expectation of connecting billions of new devices, this consumption is expected to go up \cite{gubbi2013internet}.	

One of the key application areas of IoT is envisioned to be in the smart residential buildings \cite{schatten2014smart}. Among a number of smart devices that are interconnected and installed in such buildings, the devices that support the smart grid development, such as smart energy, temperature, illumination and other types of environmental meters,  will also be present. According to Zygiaris' Smart City Reference Model \cite{zygiaris2013smart}, there are different innovation layers that can be used to describe the smart innovation and development characteristics within the smart cities. IoT should play an important role in several of those layers: from the interconnection layer with a number of sensors and actuators, through the integration layer monitoring those smart devices, to the intelligent applications layer making use of the real-time data. In China, in particular, among the largest portions of the IoT market is envisioned for the development of the smart grid \cite{shin2014socio}.

When it comes to the smart grid, there are the dimensions of demand and supply of energy that can be tackled. Tackling the demand should involve the users \cite{verbong2013smart}. While the focus on technology is still too strong and some smart grid players still perceive the users themselves as the barriers to the smart grid development process, we instead need to understand to what extent the users can act as solution to the sustainability pathway. IoT is predicted to enable transparent energy consumption information of different services in cities, from lighting, through public transport, to heating and air conditioning of public spaces \cite{zanella2014internet}. Moreover, the real-time, bidirectional connectivity between the utilities, grid and the users is suggested to lead to the improved overall efficiency of the grid \cite{yun2010research,li2011applications}. Finally, in the future smart homes, devices are expected to cooperate, actively share their energy and participate in building wide energy management systems \cite{karnouskos2010cooperative}. It is apparent how in such a context, where IoT meets the smart grid, innovative services and business applications emerge, but also security, privacy and trust gain novel importance.

\begin{svgraybox}
Through this chapter, we review the literature on applying IoT to support the design and development of the smart grid.
\end{svgraybox}
