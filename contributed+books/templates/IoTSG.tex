\section{IoT for the Smart Grid}
\label{sec:IoTSG}
The term Internet of Things (IoT) was coined by Ashton in 1999 \cite{ashton2011internet} while introducing RFID technology in the context of supply-chain management. The meaning has evolved during the past years. International Telecommunication Union (lTU) defines IoT as the worldwide network of interconnected objects uniquely addressable based on standard communication protocols. While the Internet has led to interconnection of people at an unprecedented scale, IoT is expected to interconnect also the objects around us, leading to a smart environment \cite{gubbi2013internet}.	

The IoT can be divided on the most general level into the three following layers:
\begin{itemize}
\item comprehensive sensing (perception layer),
\item reliable transmission (network layer), and
\item intelligent processing (application layer).
\end{itemize}
	
Another term that is often used interchangeably with IoT is \textit{ubiquitous computing} coined by Weiser \cite{weiser1991computer}. Ubiquitous computing is defined as ``the physical world that is richly and invisibly interwoven with sensors, actuators, displays, and computational elements, embedded seamlessly in the everyday objects of our lives, and connected through a continuous network'' (ibid.). While the IoT describes connected devices, ubiquitous computing focuses on the smart environment in which computing is pervasive. Hence, the two terms are taking a different starting stance and focus on different aspects of what is envisioned to become the Future Internet. 
The vision put forward by Weiser \cite{weiser1991computer} led to a fruitful new field within computer science (ubiquitous computing). However, 15 years later, Rogers \cite{rogers2006moving} offered a constructive critique of this vision. Namely, Rogers argued that we should switch from a computing approach to a human approach in developing the smart environment. In particular, the original vision suggested that ubiquitous computing can lead to an environment that is predicting and adapting to the people's needs, while the people were considered passive elements. Rogers argues the opposite: ``To make this happen, however, requires moving from a mindset that wants to make the  environment smart and proactive to one that enables people, themselves, to be smarter and proactive in their everyday and working practices.''

Yun and Yuxin \cite{yun2010research} discuss the possibilities of the IoT to bring about the smart grid through sensors, novel telecommunications and computing technologies. The sensors, such as smart, temperature, and illumination meters, collect energy and environmental data. They can also form a high-speed, real-time and bidirectional connection between the consumers, utilities and the electrical grid. It is envisioned that such an improved data collection and communication can support the decision making and in turn improve the overall efficiency of the grid. 

Interestingly, the technology at the heart of the IoT, the Internet itself, consumes up to 5\% of the total energy spent today in the world. Given the expectation of connecting billions of new devices, this consumption is expected to go up \cite{gubbi2013internet}.	

\begin{svgraybox}
Through this chapter, we review the literature and present our experience of adopting a socio-technical approach in designing a community-oriented smart grid user application. We discuss the challenges, implications and lessons learned from this design experience, and conclude the chapter by offering a set of good design principles which are also relevant to the design of other smart urban ecosystems. 

%The transition of energy system -to- smart energy system, passive user to active / engaged user
\end{svgraybox}
