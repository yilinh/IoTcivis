\section{Introduction}
\label{sec:intro}

The traditional science and engineering philosophy is dominated by technological determinism, the idea that technology determines societal development \cite{Mody2006,Sawyer2014,Smith1994}. Within this reductionist view, technologies are the central piece of the engineering design, where the physical world consists of a large number of diverse technological artifacts. 
The plausibility of this view is challenged by the socio-technical systems view \cite{VanDam2012} which argues that technological and social development form a ``seamless web'' where there is no room for technological determinism or the autonomy of technological systems \cite{Fleischhacker2004}. 
%
The latter view is premised on the interdependent and deeply linked relationships among the features of technological artifacts or systems and social systems (i.e. the mutual constitution) \cite{Sawyer2014}, since the man-made world also comprises a huge amount of social components -- people, communities, institutions, regulations, policies and everything that exists in the human mind -- that have shaped and been shaped by the technological components \cite{Harari2014,VanDam2012}. 
Engineering design is hence identified as a process through which technologies materialize into products, a process that substantively  shapes  and  reshapes  our  lives  and   societies and vice versa \cite{Kroes2008}. This focus on socio-technical interconnectedness becomes even  more  visible in designing new emerging technologies \cite{Kroes2008}.  

%With the world's rapid growing population\footnote{In 2014, 54\% of the world's population resides in urban areas. This figure was 30\% in 1950, and is project to be 66\% by 2050 \cite{UN2014}.}
Smart cities, for example, use technologies such as Internet-of-Things (IoT) within a large complex social context where they are embedded. The goal is to facilitate the coordination of fragmented urban sub-systems and to improve urban  life experience \cite{Glasmeier2015}. 
% 
The rise of the IoT has important socio-technical implications for people, organizations and society -- it is obvious that connecting devices is possible, we yet know little about its implications \cite{Shin2014}. A socio-technical perspective can be insightful when looking at dynamic technological development and when considering sustainable development \cite{Shin2014}. Although socio-technical systems have been studied for decades, socio-technical approaches are relatively new to the design and systems engineering communities \cite{Baxter2011,Norman2015,Sawyer2014}. Such approaches are not widely practised despite growing interests \cite{Baxter2011}. %and slow transition

\begin{svgraybox}
Through this chapter, we review the literature and present our experience of adopting a socio-technical approach in designing a community-oriented smart grid user application. We discuss the challenges, implications and lessons learned from this design experience, and conclude the chapter by offering a set of good design principles which are also relevant to the design of other smart urban ecosystems. 

%The transition of energy system -to- smart energy system, passive user to active / engaged user
\end{svgraybox}
