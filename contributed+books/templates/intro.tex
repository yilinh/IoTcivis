\section{Introduction}
\label{sec:intro}

The traditional science and engineering philosophy is dominated by technological determinism, the idea that technology determines societal development \cite{Mody2006,Sawyer2014,Smith1994}. Within this reductionist view, technologies are core to the engineering design, where the physical world consists of a large number of diverse technological artefacts. 
The plausibility of this view is challenged by the Socio-Technical Systems (STS) view \cite{VanDam2012} that argues that technological and social development form a ``seamless web'' where there is no room for technological determinism or autonomy of technological systems \cite{Fleischhacker2004}. 
%
The latter view is premised on the interdependent and deeply linked relationships among the features of technological artefacts or systems and social systems (i.e. the mutual constitution) \cite{Sawyer2014}, since the man-made world also comprises a huge number of social components -- people, communities, institutions, regulations, policies and everything that exists in the human mind -- that have shaped and been shaped by technological components \cite{Harari2014,VanDam2012}. 
In this view, engineering design is identified as a process through which technologies materialize into products, a process that substantively  shapes  and  reshapes  our  lives  and   societies and vice versa \cite{Kroes2008}. This focus on Socio-Technical (ST) interconnectedness becomes even  more  visible in new emerging technologies \cite{Kroes2008}.  

%With the world's rapid growing population\footnote{In 2014, 54\% of the world's population resides in urban areas. This figure was 30\% in 1950, and is project to be 66\% by 2050 \cite{UN2014}.}
Smart cities, for example, use technologies such as Internet-of-Things (IoT) within a large complex social context in which they are embedded to facilitate coordination of fragmented urban sub-systems and to improve urban  life experience \cite{Glasmeier2015}. 
% 
The rise of IoT has important ST implications for people, organizations and society. Although connecting devices is technically possible, little is known about the implications \cite{Shin2014}. An ST perspective can be insightful when looking at dynamic technological development and when considering sustainable development \cite{Shin2014}. Although STS have been studied for decades, ST approaches are relatively new to the design and systems engineering communities \cite{Baxter2011,Norman2015,Sawyer2014}. Such approaches are not widely practised despite growing interests \cite{Baxter2011}. %and slow transition

This chapter reviews the literature and presents our experience of adopting an ST approach in designing a community-oriented smart grid application called \textit{YouPower}. It discusses the challenges, process and outcomes of this design experience, and provides a set of lessons learned that are also deemed relevant to the design of other smart urban ecosystems. 

%The transition of energy system -to- smart energy system, passive user to active / engaged user

