%%%%%%%%%%%%%%%%%%%% author.tex %%%%%%%%%%%%%%%%%%%%%%%%%%%%%%%%%%%
%
% sample root file for your "contribution" to a contributed volume
%
% Use this file as a template for your own input.
%
%%%%%%%%%%%%%%%% Springer %%%%%%%%%%%%%%%%%%%%%%%%%%%%%%%%%%


% RECOMMENDED %%%%%%%%%%%%%%%%%%%%%%%%%%%%%%%%%%%%%%%%%%%%%%%%%%%
\documentclass[graybox]{svmult}

% choose options for [] as required from the list
% in the Reference Guide

\usepackage{mathptmx}       % selects Times Roman as basic font
\usepackage{helvet}         % selects Helvetica as sans-serif font
\usepackage{courier}        % selects Courier as typewriter font
\usepackage{type1cm}        % activate if the above 3 fonts are
                            % not available on your system
%
\usepackage{makeidx}         % allows index generation
\usepackage{graphicx}        % standard LaTeX graphics tool
                             % when including figure files
\usepackage{multicol}        % used for the two-column index
\usepackage[bottom]{footmisc}% places footnotes at page bottom

% see the list of further useful packages
% in the Reference Guide

\makeindex             % used for the subject index
                       % please use the style svind.ist with
                       % your makeindex program

%%%%%%%%%%%%%%%%%%%%%%%%%%%%%%%%%%%%%%%%%%%%%%%%%%%%%%%%%%%%%%%%%%%%%%%%%%%%%%%%%%%%%%%%%

\begin{document}

\title*{Embedding IoT in Large-scale Socio-technical Systems: A Community-Oriented Design in Future Smart Grids}
\titlerunning{Embedding IoT in Large-scale Socio-technical Systems} 

\author{Name of First Author and Name of Second Author}
% Use \authorrunning{Short Title} for an abbreviated version of
% your contribution title if the original one is too long
\institute{Name of First Author \at Name, Address of Institute, \email{name@email.address}
\and Name of Second Author \at Name, Address of Institute \email{name@email.address}}
%
% Use the package "url.sty" to avoid
% problems with special characters
% used in your e-mail or web address
%
\maketitle

\abstract*{In traditional engineering, technologies are viewed as the central piece of the engineering design, where the physical world consists of a large number of diverse technological artifacts. The real world, however, also comprises a huge amount of social components -- people, communities, institutions, regulations and everything that exists in the human mind -- that have shaped and been shaped by the technical components. Smart urban ecosystems are examples of such large-scale socio-technical systems that rely on technologies, particularly IoT, within a complex social context where the technologies are embedded. Despite that the two aspects are deeply intertwined, designing applications that embed IoT in large-scale socio-technical systems is slowly transitioning from a traditional engineering approach towards a socio-technical approach. The latter has not yet entered the mainstream of design practice. In this chapter, we present our experience of adopting a socio-technical approach in designing a community-oriented smart grid user application. The challenges, implications and lessons learned are discussed. The chapter is concluded by offering a set of good design principles derived from this experience, which are also relevant to the design of other smart urban ecosystems.}

%change the abstract above too
\abstract{In traditional engineering, technologies are viewed as the central piece of the engineering design, where the physical world consists of a large number of diverse technological artifacts. The real world, however, also comprises a huge amount of social components -- people, communities, institutions, regulations and everything that exists in the human mind -- that have shaped and been shaped by the technical components. Smart urban ecosystems are examples of such large-scale socio-technical systems that rely on technologies, particularly  Internet-of-Things (IoT), within a complex social context where the technologies are embedded. Despite that the two aspects are deeply intertwined, designing applications that embed IoT in large-scale socio-technical systems is slowly transitioning from a traditional engineering approach towards a socio-technical approach. The latter has not yet entered the mainstream of design practice. In this chapter, we present our experience of adopting a socio-technical approach in designing a community-oriented smart grid user application. The challenges, implications and lessons learned are discussed. The chapter is concluded by offering a set of good design principles derived from this experience, which are also relevant to the design of other smart urban ecosystems.}

\section{Introduction}
\label{sec:intro}

The traditional science and engineering philosophy is dominated by technological determinism, the idea that technology determines societal development \cite{Mody2006,Sawyer2014,Smith1994}. Within this reductionist view, technologies are core to the engineering design, where the physical world consists of a large number of diverse technological artefacts. 
The plausibility of this view is challenged by the Socio-Technical Systems (STS) view \cite{VanDam2012} that argues that technological and social development form a ``seamless web'' where there is no room for technological determinism or autonomy of technological systems \cite{Fleischhacker2004}. 
%
The latter view is premised on the interdependent and deeply linked relationships among the features of technological artefacts or systems and social systems (i.e. the mutual constitution) \cite{Sawyer2014}, since the man-made world also comprises a huge number of social components -- people, communities, institutions, regulations, policies and everything that exists in the human mind -- that have shaped and been shaped by technological components \cite{Harari2014,VanDam2012}. 
In this view, engineering design is identified as a process through which technologies materialize into products, a process that substantively  shapes  and  reshapes  our  lives  and   societies and vice versa \cite{Kroes2008}. This focus on Socio-Technical (ST) interconnectedness becomes even  more  visible in new emerging technologies \cite{Kroes2008}.  

%With the world's rapid growing population\footnote{In 2014, 54\% of the world's population resides in urban areas. This figure was 30\% in 1950, and is project to be 66\% by 2050 \cite{UN2014}.}
Smart cities, for example, use technologies such as Internet-of-Things (IoT) within a large complex social context in which they are embedded to facilitate coordination of fragmented urban sub-systems and to improve urban  life experience \cite{Glasmeier2015}. 
% 
The rise of IoT has important ST implications for people, organizations and society. Although connecting devices is technically possible, little is known about the implications \cite{Shin2014}. An ST perspective can be insightful when looking at dynamic technological development and when considering sustainable development \cite{Shin2014}. Although STS have been studied for decades, ST approaches are relatively new to the design and systems engineering communities \cite{Baxter2011,Norman2015,Sawyer2014}. Such approaches are not widely practised despite growing interests \cite{Baxter2011}. %and slow transition

This chapter reviews the literature and presents our experience of adopting an ST approach in designing a community-oriented smart grid application called \textit{YouPower}. It discusses the challenges, process and outcomes of this design experience, and provides a set of lessons learned that are also deemed relevant to the design of other smart urban ecosystems. 

%The transition of energy system -to- smart energy system, passive user to active / engaged user



\section{Designing in Large-scale Socio-technical Systems}
\label{sec:design}
% Always give a unique label
% and use \ref{<label>} for cross-references
% and \cite{<label>} for bibliographic references
% use \sectionmark{}
% to alter or adjust the section heading in the running head

\begin{svgraybox}
The socio-technical view can be articulated as the recognition of (1) the mutual constitution of people and technologies, (2) the contextual embeddedness of this mutuality, and (3) the importance of collective action \cite{Sawyer2014}. 
Those who hold this view examine more than just the technological system, or just the social system, or even the two side by side, but also the phenomena that emerge when the two interact \cite{Lee2001}. 
\end{svgraybox}



\begin{svgraybox}

Designing large-scale complex systems with a socio-technical view has a number of implications for (1) the formulation of the design problem, (2) the products of the design process, and (3) the design process itself (BootCamp, BC).

\runinhead{Formulation of the design problem} It is not straightforward what needs to be taken into consideration in relation to the design. What systems boundaries to choose. the question of systems boundaries is an issue for technical systems and even becomes more difficult for social technical systems
what are the issues to be addressed.[BC]

Ill-structured problem

\runinhead{Products of the design process} these not only consists of technological artifacts but also may include rules for behaviour, policies, etc. through which the designer wish to intervene in social-technical systems. what is it that we are designing? 

\runinhead{Design process} The design process can be seen as a decision-making process where the problem owners, shareholders, users, etc. participate to represent their interests. It is often conceived and implemented in participatory decision-making processes actively involving stakeholders

Large-scale socio-technical systems are often not designed as a whole but incrementally ``piece by piece'' evolving from legacy systems (BC). Designers are therefore working \textit{in} the context of some socio-technical system with the intention of changing or improving some part of that system [BC]. This means that what matters more in the design is the design process itself, more than the ``final status'' of the system \cite{Shin2014, need more ref} because the socio-technical system keeps evolving and exhibits emergent behaviour \cite{Nikolic2009}. An important  goal of the design process is to make the design (a product or system) relevant to the evolving context \cite{Shin2014, need more ref} as social and technical artifacts exist within their socio-technical context [BC]. 

\end{svgraybox}






\begin{svgraybox}
Use and combine content in:
\begin{enumerate}
\item \cite{Norman2015} (design problems in large-scale socio-technical systems) and 
\item \cite{Baxter2011} (socio-technical approach to systems engineering)
\item \cite{Whitworth2009} (four system levels of Socio-tehnical systems); 
\item \cite{Shin2014} (a very good article about IoT, socio-technical perspective )
\item see also https://medium.com/rettigs-notes/notes-on-sociotechnical-systems-design-178f161bc9e8 
\end{enumerate}
\end{svgraybox}



\section{IoT for the Smart Grid}
\label{sec:IoTSG}
The term Internet of Things (IoT) was coined by Ashton in 1999 \cite{ashton2011internet} while introducing RFID technology in the context of supply-chain management. The meaning has evolved during the past years. International Telecommunication Union (lTU) defines IoT as the worldwide network of interconnected objects uniquely addressable based on standard communication protocols. While the Internet has led to interconnection of people at an unprecedented scale, IoT is expected to interconnect also the objects around us, leading to a smart environment \cite{gubbi2013internet}.	

The IoT can be divided on the most general level into the three following layers:
\begin{itemize}
\item comprehensive sensing (perception layer),
\item reliable transmission (network layer), and
\item intelligent processing (application layer).
\end{itemize}
	
Another term that is often used interchangeably with IoT is \textit{ubiquitous computing} coined by Weiser \cite{weiser1991computer}. Ubiquitous computing is defined as ``the physical world that is richly and invisibly interwoven with sensors, actuators, displays, and computational elements, embedded seamlessly in the everyday objects of our lives, and connected through a continuous network'' (ibid.). While the IoT describes connected devices, ubiquitous computing focuses on the smart environment in which computing is pervasive. Hence, the two terms are taking a different starting stance and focus on different aspects of what is envisioned to become the Future Internet. 
The vision put forward by Weiser \cite{weiser1991computer} led to a fruitful new field within computer science (ubiquitous computing). However, 15 years later, Rogers \cite{rogers2006moving} offered a constructive critique of this vision. Namely, Rogers argued that we should switch from a computing approach to a human approach in developing the smart environment. In particular, the original vision suggested that ubiquitous computing can lead to an environment that is predicting and adapting to the people's needs, while the people were considered passive elements. Rogers argues the opposite: ``To make this happen, however, requires moving from a mindset that wants to make the  environment smart and proactive to one that enables people, themselves, to be smarter and proactive in their everyday and working practices.''

Yun and Yuxin \cite{yun2010research} discuss the possibilities of the IoT to bring about the smart grid through sensors, novel telecommunications and computing technologies. The sensors, such as smart, temperature, and illumination meters, collect energy and environmental data. They can also form a high-speed, real-time and bidirectional connection between the consumers, utilities and the electrical grid. It is envisioned that such an improved data collection and communication can support the decision making and in turn improve the overall efficiency of the grid. 

Interestingly, the technology at the heart of the IoT, the Internet itself, consumes up to 5\% of the total energy spent today in the world. Given the expectation of connecting billions of new devices, this consumption is expected to go up \cite{gubbi2013internet}.	

\begin{svgraybox}
Through this chapter, we review the literature and present our experience of adopting a socio-technical approach in designing a community-oriented smart grid user application. We discuss the challenges, implications and lessons learned from this design experience, and conclude the chapter by offering a set of good design principles which are also relevant to the design of other smart urban ecosystems. 

%The transition of energy system -to- smart energy system, passive user to active / engaged user
\end{svgraybox}


\section{CIVIS: A Community-Oriented Design in Future Smart Grids}

\begin{svgraybox}
Discuss the CIVIS project making relation to the previous theory section. 

The discussion shall not be limited to the app YouPower, but also the other efforts made around it (if they are related to the discussions in the previous section), e.g. the user stories, focus groups workshops, interviews, participatory budgeting, etc. 

Use the YouPower paper as much as possible. 
\end{svgraybox}

\subsection{Understanding and Formulation of the Design Situation}

\subsection{Products of the Design Process}

\subsection{Design Process}

or call it participatory design process? 

\section{Discussions}

\begin{svgraybox}
Lessons Learned?  / Design Guidelines? 
\end{svgraybox}


\section{Conclusions}
\begin{svgraybox}
\end{svgraybox}




\begin{figure}[b]
\sidecaption[b]
% Use the relevant command for your figure-insertion program
% to insert the figure file.
% For example, with the option graphics use
\includegraphics[scale=.65]{figure}
%
% If no graphics program available, insert a blank space i.e. use
%\picplace{5cm}{2cm} % Give the correct figure height and width in cm
%
%\caption{Please write your figure caption here}
\caption{If the width of the figure is less than 7.8 cm use the \texttt{sidecapion} command to flush the caption on the left side of the page. If the figure is positioned at the top of the page, align the sidecaption with the top of the figure -- to achieve this you simply need to use the optional argument \texttt{[t]} with the \texttt{sidecaption} command}
\label{fig:2}       % Give a unique label
\end{figure}


\subruninhead{Run-in Heading Italic Version} Use the \LaTeX\ automatism for all your cross-refer\-ences and citations as has already been described in Sect.~\ref{sec:2}\index{paragraph}.
% Use the \index{} command to code your index words
%
\begin{table}
\caption{Please write your table caption here}
\label{tab:1}       % Give a unique label
%
% Follow this input for your own table layout
%
\begin{tabular}{p{2cm}p{2.4cm}p{2cm}p{4.9cm}}
\hline\noalign{\smallskip}
Classes & Subclass & Length & Action Mechanism  \\
\noalign{\smallskip}\svhline\noalign{\smallskip}
Translation & mRNA$^a$  & 22 (19--25) & Translation repression, mRNA cleavage\\
Translation & mRNA cleavage & 21 & mRNA cleavage\\
Translation & mRNA  & 21--22 & mRNA cleavage\\
Translation & mRNA  & 24--26 & Histone and DNA Modification\\
\noalign{\smallskip}\hline\noalign{\smallskip}
\end{tabular}
$^a$ Table foot note (with superscript)
\end{table}

\begin{description}[Type 1]
\item[Type 1]{That addresses central themes pertainng to migration, health, and disease. Blablabla}
\item[Type 2]{That addresses central themes pertainng to migration, health, and disease. Blablabla}
\end{description}





\begin{acknowledgement}
If you want to include acknowledgments of assistance and the like at the end of an individual chapter please use the \verb|acknowledgement| environment -- it will automatically render Springer's preferred layout.
\end{acknowledgement}
%
%\section*{Appendix}
%\addcontentsline{toc}{section}{Appendix}
%


\input{referenc}
\end{document}
