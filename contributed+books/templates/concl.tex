
\section{Discussion and Conclusion}

Collaboratively designed with the stakeholders from different pilot sites, the main outcomes of CIVIS addressed the goals and context of the project at different ST levels. They include the CIVIS platform that consisted of YouPower, an open source social smart grid application, and the corresponding hardware and software installation for energy data collection at participating households from the pilot sites. The   deployment was accompanied by community engagement approaches to ensure that the stakeholders were well aware of the key issues and results the project was aiming at, and to develop positive attitude and encourage active participation. 
% 

At the Italian pilot sites, self-consumption of local renewable production was promoted at household and consortium levels, while at the Swedish pilot sites, knowledge sharing about housing cooperatives energy management practices was supported among cooperatives' board members and across different cooperatives. To bridge the attitude-behaviour gap of people's environmental values (and attitudes) and their actual behaviour in energy consumption \cite{Kollmuss2002,Schultz2002,Schultz2014}, the platform also provided a set of features that could facilitate users' behaviour change process towards sustainable  consumption that was implementable in their daily life along their existing practices. 
% 
A number of lessons learned from the CIVIS design experience that could also be relevant to the design of other smart urban ecosystems beyond the particular case of CIVIS project are discussed below. 
% 

First, despite the many advantages already discussed previously, implementing a collaborative participatory design process is highly challenging in practice with an interdisciplinary team in an international setting. The design and development team, together with stakeholders involved, have various professional and cultural backgrounds, possibly speak diverse languages, hold disparate values and principles, work in different styles, not to mention the personal and organizational interests they may withhold. Misunderstandings on terminologies,  methodologies and actions may go unnoticed and accumulate until it is very challenging or even critical to mediate the diverging opinions. The full awareness of such issues, frequent and efficient communications,  positive and constructive attitude, plus open-mindedness are the keys to make the development process effective and enjoyable.
% 

Second, the relevance, importance and challenge of setting up an engagement strategy or change process for the potential users of the new or modified system should not be underestimated.
Engaging people in changing behaviour has much to do with understanding local contexts, people's heterogeneous attitudes, and local cultures. It also needs careful planning and execution.
Develop a clear engagement strategy starting from the beginning of the project and let the professionals with the proper skills in this area to interact with the stakeholders. 
% 

Third, with respect to STS design and engagement strategies, users and other stakeholders should be provided with accurate and actionable information about how to achieve target behaviour. At the CIVIS pilot sites, for example,  people expressed the desire and need to want to do more for a sustainable future. They liked the idea of receiving relevant and contextual suggestions and tips for action. Given the heterogeneity of potential stakeholder groups, understanding them and their interests and needs remains a crucial and challenging part of design that requires careful confrontation with stakeholders directly.
% 

Fourth, consumers' intrinsic motivation for engagement needs to be fostered. Users need to be allowed to freely practice and adapt their course of action. This facilitates the sense of competence and autonomy that promotes and enhances motivation for behaviour change \cite{Ryan2000}. 
For example, people in the pilot sites are skeptical about how much monetary gains they can actually have by using less energy in households, but they are driven by intrinsic motives as well as altruistic and environmental values for energy saving. The social and community-oriented features as those designed in the CIVIS project  articulate those values. 

With the explosive growth of smart devices and smart everything, the coming wave of  IoT and the hyperconneted world will soon bring the society into a smart environment where computing is pervasive \cite{gubbi2013internet,Shin2014}. Will this smartness bring its inventors and the natural world into a sustainable future? This chapter advocates the potential fruitfulness of IoT and smart urban ecosystems that do not mainly rely on the technological side. Designers and engineers need to indispensably take a human-centred ST approach in developing a smart sustainable future.   


% is designed and developed as a set of open source packages that are composable and extensible (under Apache v.2 license) to different needs related to energy conservation and load shifting interventions. The development was completed by June 2016, and different parts were deployed to the test sites in Stockholm and Trento respectively. 
%The initial deployment of YouPower shows that household engagement varies significantly between test sites and among participants. The main reason for this can probably be found in the different characteristics of the communities: elderly people and parents with older children (mainly in the Italian test site) in particular participated enthusiastically, while a much smaller number of families with young children (mainly in the Swedish test site) participated actively in the project. Preliminary results seem to suggest that engagement can be driven not only by individuals' pre-existing motivations (e.g., financial or environmental) but also by households' experiences and interactions once they start actively using an application such as YouPower. We conjecture that more engagement of users will lead to more energy reductions. % and ultimately a more sustainable world. More research is needed to support this conclusion, which is left for future work. 
%While the initial deployment of YouPower shows that household engagement varies significantly between the two test sites and among participants, preliminary results do suggest that a community-oriented intervention, such as YouPower, increases user engagement significantly.
%, particularly for elderly people and parents with older children. 
%The results suggest that engagement can be driven not only by individuals' pre-existing motivations (e.g., financial or environmental) but also by households' experiences and interactions once they start actively using an application such as YouPower. We conjecture that more user engagement will lead to more energy reductions and better load-shifting results. % and ultimately a more sustainable world. 
% More research is needed to support this conclusion, which is left for future work.
%An extended data collection period for the households' energy consumption data at the test sites is needed (to obtain an annual energy consumption pattern during the intervention period to be compared to pre-intervention period data) for the research to draw a conclusion on the effect of the intervention on energy consumption behaviour in the test sites. This effort is left for future work.
